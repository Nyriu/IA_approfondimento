%!TEX TS-program = pdflatex
%!TEX root = main.tex
%!TEX encoding = UTF-8 Unicode


\section{RL}
Il \emph{Reinforcement Learning} (RL), assieme al \emph{Supervised Learning} ed al \emph{Unsupervised Learning}, è il terzo paradigma di apprendimento autonomo.
Gli ultimi due paradigmi sfruttano un dataset, rispettivamente con e senza label, per portare a termine un specifico task oppure generare nuova informazione.
Nel caso del RL il dataset viene sostituito con un ambiente (\emph{environment}) nel quale il modello può eseguire delle azioni e osservarne le conseguenze, quindi come l'ambiente viene modificato dall'azione appena compiuta.
In questo paradigma il modello viene anche chiamato ``agente'' e l'elenco, discreto oppure continuo, delle azioni eseguibili prende il nome di \emph{action space}.
L'agente impara grazie ad una ricompensa (\emph{reward}), positiva o negativa, associata al risultato delle sue azioni.
Il suo obiettivo è quello di massimizzare la somma dei \emph{reward} sul lungo termine, quindi si vuole che scopra e adotti una strategia (\emph{policy}) efficace nell'ambiente considerato.
Poiché è necessario svolgere numerose iterazioni del tipo \emph{trial-and-error} e considerato che solitamente un fallimento corrisponde anche ad una grande acquisizione di informazione, bisogna creare degli \emph{environment} virtuali che siano il più vicino possibile al dominio di applicazione finale e che permettano un'esecuzione rapida e senza costi aggiuntivi.
%
%\todo{ampliare intro?} % TODO
%\todo{esempi con ref?} % TODO

Riprendendo quanto illustrato in \cite{MIT_RL} ed in \cite{Simple_RL}, una prima modellazione sfrutta una funzione che valuta la qualità dell'azione svolta 
$$
Q(s_t, a_t) = \E [ R_t | s_t, a_t]
$$
in cui 
$s_t$ è lo stato corrente cioè come l'ambiente si presenta all'agente,
$a_t$ è l'azione che l'agente svolge all'istante $t$,
mentre a destra dell'uguale si ha il valore atteso del \emph{reward} totale $R_t$ che l'agente potrà ricevere in futuro, quindi negli $s_{t+i}$ con $i=1,2,\dots$ se esegue $a_t$ in $s_t$.
Calcolare $R_t$ risulta critico perché la sua naturale definizione è
$$
R_t = \sum_{i=t}^{\inf} \gamma^i r_i = 
\gamma^{t+1} r_{t+1} + 
\gamma^{t+2} r_{t+2} + 
\gamma^{t+3} r_{t+3} + \dots
\gamma^{t+n} r_{t+n} + \dots
$$
in cui $0 < \gamma < 1$  viene chiamato \emph{discount factor} ed indica la \emph{greediness} del modello. %\todo{spiegare meglio in caso}
In questa casistica risulta utile approssimare $Q(s,a)$ con una rete neurale, in questo modo si evita di dover fissare a mano un \emph{hyperparameter} come, ad esempio,  il numero di somme da effettuare per approssimare $R_t$.
L'idea è fornire alla rete, detta anche \emph{Q-Network} \cite{DQN}, lo stato corrente ed ogni possibile azione lecita e successivamente scegliere l'azione a cui la rete assegna il valore più alto.
$$
argmax_a Q(s_t, a) %\quad\textrm{con $Q(\cdot)$ approssimata da una rete}
$$
Procedendo in questo modo per ogni stato che si incontra $s_{t+i}$ con $i=1,2,\dots$ è possibile seguire una \emph{policy} $\pi(s)$ ottimale ad ogni passo.
La \emph{policy} $\pi(s)$ è una politica di scelta che associa uno stato ad un'azione, con lo scopo di ottenere un buon guadagno sul lungo termine.
Si trova una strategia ottimale quando ogni scelta fatta seguendo la $\pi$ è ottimale.

%\todo[inline]{qui mettere loss function e spiegarla?}

L'approccio appena presentato richiede che l'\emph{action space} sia discreto e di dimensione ridotta, altrimenti sarebbe impossibile o molto costoso iterare su tutte le azioni per selezionarne la migliore.
Per ovviare a questo problema si possono utilizzare modelli che provano ad ottimizzare direttamente la \emph{policy} $\pi(s)$, questi modelli prendono il nome di \emph{Policy Learning} e vengono allenati tramite il \emph{Policy Gradient}. %\todo{cite paper?}
In questo modo è possibile ottenere direttamente l'azione migliore a partire soltanto dallo stato.
L'idea principale è approssimare $\pi$ con una distribuzione di probabilità, in questo modo risulta naturale utilizzare un \emph{action space} continuo, inoltre si può ottenere una strategia non deterministica, quindi più flessibile e con maggiori capacità esplorative durante il training.
Il non determinismo è introdotto quando, per scegliere l'azione da eseguire, se ne estrae una dalla distribuzione di probabilità data dalla \emph{policy}.
In formule, dato uno stato $s$ l'azione $a$ verrà estratta secondo:
%$$
%a = 
%\pi(s) \sim P(a|s) \quad\textrm{in cui } \sum_{a_i \in A} P(a_i | s) = 1 \quad\textrm{con $A$ \emph{action space}}
%$$
$$
a = 
\pi(s) \sim P(a|s) \quad\textrm{in cui } \int_{a = - \inf}^{\inf} P(a | s) = 1
$$
Poiché $P(a|s)  = \mathcal{N}(\mu,\sigma ^2)$ si può fare in modo che la rete dia in output direttamente il vettore della medie $\mu$ e il vettore delle varianze $\sigma^2$.
Questo è utile perché, come succedeva per i VAE, risolve il problema della differenziabilità dell'operazione di \emph{sampling} di $z$.
L'allenamento incomincia con l'inizializzazione dell'agente e dell'ambiente, poi l'agente segue la sua \emph{policy} corrente fino a terminazione mentre tutti gli stati, le azioni e i \emph{reward} vengono registrati.
Infine si aumenta la probabilità delle azioni che hanno portato a \emph{reward} positivi e si decrementa la probabilità delle azioni con \emph{reward} negativo, fatto ciò si effettua nuovamente l'inizializzazione in modo da valutare la \emph{policy} aggiornata. 
Notare che definire il concetto di ``terminazione'' non è sempre ovvio e può corrispondere, ad esempio, all'esecuzione di un numero limitato di azioni oppure al ricevimento di un grande \emph{reward} negativo.
L'aggiornamento dei pesi della rete, quindi della \emph{policy}, viene effettuato tramite \emph{gradient descent} usando la \emph{loss}
$$
loss = - log P(a_t | s_t ) R_t
$$
Il logaritmo indica la probabilità logaritmica con cui $a_t$ viene scelta allo stato $s_t$ mentre $R_t$ è il valore atteso del \emph{reward} totale che si guadagnerà.
Se $R_t$ è positivo allora converrà aumentare la probabilità di selezionare $a_t$, viceversa se è negativo conviene diminuire la probabilità.
Il \emph{Policy Gradient} è il gradiente relativo a questa particolare \emph{loss}.
Rimane comunque critico il calcolo di $R_t$.
Usando i \emph{Monte Carlo Rollouts} è possibile approssimare questo valore.
In particolare, a partire dallo stato corrente, vengono generate (``srotolate'' da \emph{rollout}) svariate strategie seguendo la \emph{policy} corrente.
Così facendo si può stimare un \emph{range} di valori raggiungibili ed effettuare una media di questi valori.
L'algoritmo appena descritto prende il nome di REINFORCE e può essere diviso in quattro passaggi:
\begin{enumerate}
  \item dallo stato corrente $s_t$ si esegue varie volte $\pi(s_t)$ fino a terminazione;
  \item si effettua la media dei \emph{reward} ottenuti da ogni esecuzione;
  \item si usa il valore ottenuto per aggiornare la rete;
  \item si estrae un azione $a_t$ dalla distribuzione $\pi(s_t)$ e si ripete dal punto 1.
\end{enumerate}
Nell'ambito del \emph{Reinforcement Learning} spesso i termini \emph{explore} ed \emph{exploit} vengono usati per indicare due momenti diversi del train della rete.
Una rete viene usata in modalità \emph{explore} quando si vuole che sperimenti strategie nuove.
In questa modalità si cerca di trovare soluzioni alternative, sperando che siano migliori di quella corrente.
La modalità \emph{exploit}, invece, viene usata quando si vuole sfruttare le conoscenze apprese e portare a termine il compito secondo quella che è ritenuta la strategia migliore.
%La transizione da una modalità all'altra è controllata da un valore detto ``di temperatura'' $\alpha$ 

Quasi obbligatorio citare i risultati ottenuti da AlphaGO \cite{AlphaGO} ed AlphaZero \cite{AlphaZero} attraverso il RL.
Particolare interessante il fatto che sul primo modello è stato effettuato un \emph{pre-train} per simulare il comportamento umano, mentre il secondo è stato allenato esclusivamente (da zero) facendo giocare due agenti RL avversari nello stesso ambiente (la partita).















%\clearpage
%\todo[inline]{TODO più matematicoso con spiegazione calcolo $R_t$}
%\todo[inline]{TODO differenza tra esplorare e exploit}
%\todo[inline]{TODO esempio Alpha GO, Alpha Zero e Hide and Seek}

