%!TEX TS-program = pdflatex
%!TEX root = main.tex
%!TEX encoding = UTF-8 Unicode

\section{Introduzione}

Questo approfondimento ha come oggetto le tecniche di \emph{Machine Learning} (ML) \emph{Generative Adversarial Network} (GAN) e \emph{Reinforcement Learning} (RL).
In  particolare nelle  prima sezione si  esplorano le tecniche generative, definendo prima i \emph{Variational  Autoencoder} e  poi le GAN.
Nella seconda sezione si definiscono i concetti alla base  del  \emph{Reinforcement Learning} e come le tecniche di ML possono essere sfruttate in questo ambito.
La terza  sezione illustra come si può combinare l'architettura GAN con tecniche prese dal mondo del RL per generare testi sintetici che siano verosimili.
Nello specifico si analizzano l'architettura SeqGAN \cite{SeqGAN} e la sua evoluzione LeakGAN \cite{LeakGAN}.
L'ultima sezione illustra brevemente un'implementazione delle LeakGAN scritta dagli stessi autori dell'articolo che la introduce \cite{LeakGAN}.
