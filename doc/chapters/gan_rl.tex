%!TEX TS-program = pdflatex
%!TEX root = main.tex
%!TEX encoding = UTF-8 Unicode


\section{GAN+RL per testi}
\todo[inline]{fare intro RNN? LSTM?}
In questa sezione vengono illustrati due modelli capaci di generare testi sintetici sfruttando un'architettura GAN in cui $G$ viene allenato attraverso \emph{reinforcement learning} .
Il primo modello, chiamato SeqGAN, è stato presentato in \cite{SeqGAN} ed illustrato anche in \cite{GAN_for_text}; il secondo è evoluzione del primo, permette di generare testi più lunghi, prende il nome di LeakGAN ed è descritto in \cite{LeakGAN}.
\todo[inline]{Sono particolarmente interessanti perché con SeqGAN si risolve} % TODO se serve

\subsection{SeqGAN}
Come riportato nell'introduzione dell'articolo \cite{SeqGAN}, per generare frasi che siano verosimili è necessario allenare un discriminatore che valuti frasi intere e che assegni a queste un punteggio.
Purtroppo ciò rende molto difficile allenare il generatore, perché non è possible determinare se un punteggio basso corrisponde all'intera struttura della frase oppure soltanto ad una o poche parole.
La problematica è ancora più evidente nel caso in cui il generatore è una RNN \todo{mai introdotte per ora, TODO da fare sopra}
rendendo difficile, ad esempio aggiornare efficacemente il modo con cui vengono create le parti iniziali di frasi.

Le SeqGAN affrontano il problema in un modo molto interessante: se si considera il punteggio che $D$ fornisce alle frasi come \emph{reward} per $G$ e se questo utilizza come stato la frase generata fino ad ora e come azione la scelta della parola successiva, allora è possibile sfruttare il \emph{Policy Gradient} sul generatore.
Di fondamentale importanza la \emph{Monte Carlo Search} che viene effettuata per valutare la bontà di frasi incomplete, così da alterare efficacemente la distribuzione della parola che ancora deve essere scelta: 
durante la generazione di una frase, $G$ non può ricevere una valutazione da $D$ perché il discriminatore è in grado di valutare soltanto frasi intere % TODO fare osservazione? % (una frase incompleta sarà sempre poco verosimile 
quindi vengono generate $N$ frasi con prefisso la frase generata fino ad ora.
Si sfrutta poi $D$ per valutare tutte le $N$ frasi e si effettua una media dei \emph{reward} ottenuti, così si ottiene il valore atteso della bontà della frase che si sta generando.
Ci si riferisce a questo furbo accorgimento come \emph{Monte Carlo state-action search}.

Riprendendo i formalismi usati nell'articolo si ha:
\begin{itemize}
  \item un modello generativo $G_\theta$, $\theta$ indica i parametri interni, in grado di generare sequenze $Y_{1:T} = ( y_1, \dots , y_t, \dots , y_T)$ con gli $y_t$ appartenenti all'insieme dei token validi $\T$;
  \item al tempo $t$ lo stato $s$ equivale ai token prodotti fino ad ora $(y_1, \dots , y_{t-1})$ mentre l'azione $a$ è il prossimo token da selezionare $y_t$;
  \item con $G_\theta (y_t | Y_{1 : t-1} )$ si indica il modello non deterministico descritto.

  \item Il modello discriminativo $D_\phi$ con parametri $\phi$ in grado di fornire la probabilità $D_\phi ( Y_{1:T})$ che $Y_{1:T}$ sia stato estratto dai dati reali.
\end{itemize}
% TODO \todo[inline]{qui immagine rete?}

Prima di continuare con la \emph{loss function} e la formulazione della \emph{Monte Carlo search}, 
va sottolineato che il modello RNN è leggermente diverso da quello classico, infatti ad ogni passo la rete prende in input il token generato al passo precedente anziché riceverlo dall'esterno.
Si può quindi dire che assomigli ai modelli RNN usati come decoder durante la traduzione di testi, nei quali lo stato interno e l'ultima parola tradotta vengono utilizzati per aggiornare lo stato e generare la parola successiva.
\todo{add ref to \url{https://arxiv.org/pdf/1506.03099v1.pdf}}
%Notare anche che la codifica interna 
Si fa presente che lo stato di partenza è TODO e l'input è parte da \todo[inline]{partenza con $s_0$ come viene fatta? $h_0$ com'è?}
Si può notare come questa sia la rivisitazione del tipico input randomico $z$ di una generica GAN.

L'obiettivo del generatore $ G_\theta $ è quello di produrre una sequenza a partire dallo stato $s_0$ che massimizzi il \emph{reward} totale, in formule:
$$
J(\theta) = \E[R_t | s_0, \theta ] =
\sum_{y_1 \in \T}
G_\theta ( y_1 | s_0) \cdot
Q_{D_\phi}^{G_\theta} ( s_0, y_1)
$$
in cui $Q_{D_\phi}^{G_\theta} ( s, a)$ è la funzione che indica il \emph{reward} accumulabile eseguendo l'azione $a$ allo stato $s$ e seguendo la \emph{policy} $G_\theta$ nei passi successivi.
Questa funzione dovrà necessariamente essere stimata, perché sappiamo che $D_\phi$ non può essere sfruttato su sequenze incomplete.
Quindi si utilizza una \emph{$N$-Monte Carlo Search} per stimare $N$ volte i $T-t$ token mancanti
$$
\{ 
  Y_{1:T}^1,
  \dots,
  Y_{1:T}^N
\}
=
MC (Y_{1:t}; N)
$$
Gli $ Y_{t+1:T}^n $ con cui si completa la sequenza di partenza sono campionati usando la stessa \emph{policy} $G_\theta$.
Quindi la stima del \emph{reward} atteso è data da
%$$
%Q_{D_\phi}^{G_\theta} ( s_0, y_1) TODO
%=
%\left\{
%\begin{array}{lr}
%  x(n), & \text{for } 0\leq n\leq 1\\
%\end{array}
%\right\
%$$

$$
Q_{D_\phi}^{G_\theta}
( s = Y_{1:t-1} ,
a = y_t)
=
\left\{\begin{array}{lr}
    \dfrac{1}{N} \sum_{n=1}^N D_\phi (Y_{1:T}^n) ,\; Y_{1:T}^n \in MC(Y_{1:t};N)
      & \textrm{for } t < T \\

    D_\phi (Y_{1:t}) & \textrm{for } t = T

\end{array}\right.
$$
%Abbiamo visto come $J(\theta)$ può essere calcolata e sappiamo che 

Per quanto riguarda il discriminatore $D_\phi$ viene specificato che l'aggiornamento dei suoi parametri $\phi$ viene effettuato solo quando il generatore ha creato un numero sufficiente di sequenze.
In questo modo è possibile avere un discriminatore che si adatta e migliora assieme al generatore, pur lasciandogli il tempo di perfezionarsi.
In formule $D_\phi$ viene allenato secondo:
$$
min_\phi
- \E_{Y \sim p_{real}} [ log D_\phi (Y) ]
- \E_{Y \sim G_\theta} [ log ( 1 - D_\phi(Y))]
$$
\todo[inline]{riportare schema algoritmo}

È molto importante sottolineare il \emph{pre-train} effettuato per inizializzare la SeqGAN con alcune conoscenze basilari.
In questo modo $G$ e $D$ saranno già capaci di svolgere i loro compiti e potranno migliorarsi più efficacemente.
Il \emph{pre-train} del generatore viene effettuato usando la \emph{Maximum Likelihood Estimation}(MLE) sul dataset di sequenze reali, $G$ tenterà quindi di imitare nel miglior modo possibile la distribuzione dei token delle sequenze date.
Mentre $D$ viene allenato come un classificatore attraverso la \emph{cross entropy loss} \todo{sigla? Formula?}
su dati reali e dati generati dal $G$ appena creato.
Ovviamente il \emph{train} di $D$ viene sempre effettuato su un insieme di sequenze per metà generato e per metà reale, così da non introdurre sbilanciamenti nelle probabilità.
\todo[inline]{già che si parla di $D$ cfr. a paper con CNN?}

In \cite{SeqGAN} vengono utilizzate anche tecniche come \emph{Dropout} e \emph{L2 regularization} per evitare l' \emph{over-fitting.} \todo{ripassare L2}
Nell'articolo è possibile trovare una valutazione dettagliata delle prestazioni delle SeqGAN rispetto ad altri modelli e su tre casi d'uso differenti.


\subsection{LeakGAN}
Le LeakGAN sono state create per far fronte alla principale debolezza delle SeqGAN, ossia la difficoltà nel generare sequenze lunghe che siano convincenti.
Se gli esperimenti delle SeqGAN mostravano affidabilità con sequenze fino a 20 token, le LeakGAN riescono a raggiungere lunghezze di 40 token, pur mantenendo coerenza e verosimiglianza.
Queste reti vengono presentate in \cite{LeakGAN} e si differenziano dalle precedenti per due motivi:
\begin{itemize}
  \item si introduce una ``perdita'' (\emph{leak}) di informazione dal discriminatore al generatore.
    Le \emph{feature} che il primo estrae e su cui poi baserà la valutazione vengono fornite al secondo in modo da ricevere un'informazione molto più ricca di un semplice punteggio; 
  \item si introduce anche un nuovo modulo all'interno del generatore in modo da elaborare l'informazione che giunge da $D$ ed utilizzarla per poi decidere il token successivo.
\end{itemize}

\todo[inline]{inserire immagine rete TODO}

Va subito fatto notare come la seconda modifica renda il generatore un generatore gerarchico, quindi composto da sottomoduli con specifici compiti.
È altrettanto importante sottolineare che i sotto-compiti che il Manager richiede al Worker sono auto-determinati, infatti il Manger è in grado di richiedere punteggiature e particolari strutture.

La \emph{Linear Projection} presente nello schema effettua una trasformazione lineare $\psi$, con pesi $W_\psi$, su un numero $c$ di goal $g_t$ recenti, così da generare un vettore $w_t$ di dimensione adeguata per l'esecuzione del prodotto con $O_t$.
I formule si ha
\begin{align}
  w_t &= \psi \left( \sum_{i=1}^{c} g_{t-1} \right)
  \\
  O_t, h_t &= Worker(x_t, h_{t-1}; \theta)
  \\
  G_\theta(\cdot|s_t) &= softmax ( O_t \cdot w_t / \alpha)
\end{align}
in cui $h_t$ è \emph{hidden state} del Worker, mentre $\alpha$ viene usato per bilanciare esplorazione e sfruttamento (\emph{exploration and exploitation}).
In generale avrà un valore alto durante il \emph{training} per favorire l'esplorazione, avrà invece un valore basso durante la generazione di quelle sequenze che poi verranno usate per allenare $D$.

\todo{sezione su train G? Rescaled $R_t$?}


Un'altra importante modifica riguarda l'\emph{interleaved training}: si alternano allenamento tramite GAN ed allenamento tramite metodo supervisionato (MLE), anziché effettuare soltanto GAN dopo il \emph{pre-train}.
Questo evita il \emph{mode collapse} obbligando $G$ a rimanenre aderente alla vera dispribuzione degli esempi reali 

TODO qua


