%!TEX TS-program = pdflatex
%!TEX root = main.tex
%!TEX encoding = UTF-8 Unicode
\def \NUMSLIDES {123}

\section{SET DI SLIDE ESEMPIO}

\slide{23}
\lipsum[1]


\slide{26}
\exercise{1}
\lipsum[1]
\exercise{2}
\lipsum[2]

\slide{60}


\begin{table}[!htb]
  \caption{Caption generale}
  \begin{adjustwidth}{0cm}{} % dx / sx

    \begin{minipage}{.5\linewidth}
      \centering
      \vspace{-2.2em} % alzo / abbasso
      \begin{tabular}{l|l|l}
        Molte    & Robe   & Belle    \\ \hline
        Mica     & Tanto  & \dots    \\
      \end{tabular}

    \end{minipage}
    \begin{minipage}{.5\linewidth}
      \centering
      \begin{tabular}{l|l|l}
        Molte    & Robe   & Belle    \\ \hline
        Mica     & Tanto  & \dots    \\
        Molte    & Robe   & Belle    \\ \hline
        Mica     & Tanto  & \dots    \\
      \end{tabular}

    \end{minipage} 
  \end{adjustwidth}
\end{table}


\begin{figure}[!htb]
  \centering
  \begin{tabular}[t]{cc}
    \begin{subfigure}{0.4\textwidth}
      \centering
      \includegraphics[width=0.9\linewidth]{example-image-a}
      \captionsetup{width=0.9\linewidth}
      \caption{Img dx}
      %\label{fig:}
    \end{subfigure}
    &
    \begin{subfigure}{0.4\textwidth}
      \centering
      \includegraphics[width=0.9\textwidth]{example-image-a}
      \captionsetup{width=.9\linewidth}
      \caption{Img dx}
      %\label{fig:}
    \end{subfigure}
  \end{tabular}
  \caption{Caption generale}
\end{figure}
